\newpage
\section*{Part 1}
\subsection*{Exercise 1}

\subsubsection*{Assumptions and System Dynamics}
The system consists of a UGV and UAV modeled as nonlinear kinematic systems.

\noindent\textit{UGV Dynamics:}
\[
\dot{\xi}_g = v_g \cos\theta_g, \quad \dot{\eta}_g = v_g \sin\theta_g, \quad \dot{\theta}_g = \frac{v_g}{L} \tan\phi_g
\]

\noindent\textit{UAV Dynamics:}
\[
\dot{\xi}_a = v_a \cos\theta_a, \quad \dot{\eta}_a = v_a \sin\theta_a, \quad \dot{\theta}_a = \omega_a
\]

\noindent The state vector is:
\[
x = [\xi_g, \eta_g, \theta_g, \xi_a, \eta_a, \theta_a]^T, \quad
u = [v_g, \phi_g, v_a, \omega_a]^T
\]

\subsubsection*{State Transition Matrix (\( A \))}
The matrix \( A = \frac{\partial f(x, u)}{\partial x} \) represents the partial derivatives of the dynamics with respect to the state variables.

\noindent\textbf{UGV Contributions:}
\[
\hat{A}_{\text{UGV}} = 
\begin{bmatrix}
0 & 0 & -v_g \sin\theta_g \\
0 & 0 & v_g \cos\theta_g \\
0 & 0 & 0
\end{bmatrix}
\]

\noindent\textbf{UAV Contributions:}
\[
\hat{A}_{\text{UAV}} = 
\begin{bmatrix}
0 & 0 & -v_a \sin\theta_a \\
0 & 0 & v_a \cos\theta_a \\
0 & 0 & 0
\end{bmatrix}
\]

\noindent\textbf{Combined System:}
\[
A_{\text{comb}} =
\begin{bmatrix}
\hat{A}_{\text{UGV}} & 0 \\
0 & \hat{A}_{\text{UAV}}
\end{bmatrix}
=
\begin{bmatrix}
0 & 0 & -v_g \sin\theta_g & 0 & 0 & 0 \\
0 & 0 & v_g \cos\theta_g & 0 & 0 & 0 \\
0 & 0 & 0 & 0 & 0 & 0 \\
0 & 0 & 0 & 0 & 0 & -v_a \sin\theta_a \\
0 & 0 & 0 & 0 & 0 & v_a \cos\theta_a \\
0 & 0 & 0 & 0 & 0 & 0
\end{bmatrix}
\]

\subsubsection*{Control Input Matrix (\( B \))}
The matrix \( B = \frac{\partial f(x, u)}{\partial u} \) represents the partial derivatives of the dynamics with respect to the control inputs.

\noindent\textbf{UGV Contributions:}
\[
\hat{B}_{\text{UGV}} =
\begin{bmatrix}
\cos\theta_g & 0 \\
\sin\theta_g & 0 \\
\frac{\tan\phi_g}{L} & \frac{v_g}{L \cos^2\phi_g}
\end{bmatrix}
\]

\noindent\textbf{UAV Contributions:}
\[
\hat{B}_{\text{UAV}} =
\begin{bmatrix}
\cos\theta_a & 0 \\
\sin\theta_a & 0 \\
0 & 1
\end{bmatrix}
\]

\noindent\textbf{Combined System:}
\[
B_{\text{comb}} =
\begin{bmatrix}
\hat{B}_{\text{UGV}} & 0 \\
0 & \hat{B}_{\text{UAV}}
\end{bmatrix}
=
\begin{bmatrix}
\cos\theta_g & 0 & 0 & 0 \\
\sin\theta_g & 0 & 0 & 0 \\
\frac{\tan\phi_g}{L} & \frac{v_g}{L \cos^2\phi_g} & 0 & 0 \\
0 & 0 & \cos\theta_a & 0 \\
0 & 0 & \sin\theta_a & 0 \\
0 & 0 & 0 & 1
\end{bmatrix}
\]

\subsubsection*{Measurement Jacobians (\( H \))}

For relative range (\( r \)) and azimuth (\( \theta_r \)), the measurement Jacobian \( H \) combines the derivatives of measurement equations with respect to the states:
\[
H =
\begin{bmatrix}
\frac{\partial \theta_r}{\partial x} \\
\frac{\partial r}{\partial x}
\end{bmatrix}
\]
Key terms derived for \( \frac{\partial \theta_r}{\partial x} \) and \( \frac{\partial r}{\partial x} \) were:
\[
\frac{\partial \theta_r}{\partial \xi_g} = \frac{\eta_a - \eta_g}{r^2}, \quad
\frac{\partial \theta_r}{\partial \eta_g} = \frac{\xi_g - \xi_a}{r^2}, \quad
\frac{\partial r}{\partial \xi_g} = \frac{\xi_g - \xi_a}{r}, \quad
\frac{\partial r}{\partial \eta_g} = \frac{\eta_g - \eta_a}{r}
\]

\subsubsection*{Summary of Jacobians}
The Jacobian matrices computed for the dynamics and measurement models are:
\begin{itemize}
    \item \( A \): \( 6 \times 6 \) state transition matrix.
    \item \( B \): \( 6 \times 4 \) control input matrix.
    \item \( H \): Measurement Jacobian derived for range and azimuth.
\end{itemize}

\subsection*{Exercise 2}


\subsubsection*{1. System Dynamics}
The continuous-time (CT) nonlinear dynamics of the system are:
\begin{itemize}
    \item \textbf{UGV Dynamics:}
    \[
    \dot{\xi}_g = v_g \cos\theta_g, \quad \dot{\eta}_g = v_g \sin\theta_g, \quad \dot{\theta}_g = \frac{v_g}{L} \tan\phi_g
    \]
    \item \textbf{UAV Dynamics:}
    \[
    \dot{\xi}_a = v_a \cos\theta_a, \quad \dot{\eta}_a = v_a \sin\theta_a, \quad \dot{\theta}_a = \omega_a
    \]
\end{itemize}

\noindent The state vector is:
\[
x = [\xi_g, \eta_g, \theta_g, \xi_a, \eta_a, \theta_a]^T, \quad
u = [v_g, \phi_g, v_a, \omega_a]^T
\]
The nominal operating point is:
\[
x_0 = [\xi_{g,0}, \eta_{g,0}, \theta_{g,0}, \xi_{a,0}, \eta_{a,0}, \theta_{a,0}]^T, \quad
u_0 = [v_{g,0}, \phi_{g,0}, v_{a,0}, \omega_{a,0}]^T
\]

\subsubsection*{2. Linearization}
The system is linearized around the nominal operating point:
\[
A = \frac{\partial f(x, u)}{\partial x} \bigg|_{x_0, u_0}, \quad
B = \frac{\partial f(x, u)}{\partial u} \bigg|_{x_0, u_0}
\]
From Question 1, the linearized matrices are:
\[
A =
\begin{bmatrix}
0 & 0 & -v_g \sin\theta_g & 0 & 0 & 0 \\
0 & 0 & v_g \cos\theta_g & 0 & 0 & 0 \\
0 & 0 & 0 & 0 & 0 & 0 \\
0 & 0 & 0 & 0 & 0 & -v_a \sin\theta_a \\
0 & 0 & 0 & 0 & 0 & v_a \cos\theta_a \\
0 & 0 & 0 & 0 & 0 & 0
\end{bmatrix}, \quad
B =
\begin{bmatrix}
\cos\theta_g & 0 & 0 & 0 \\
\sin\theta_g & 0 & 0 & 0 \\
\frac{\tan\phi_g}{L} & \frac{v_g}{L \cos^2\phi_g} & 0 & 0 \\
0 & 0 & \cos\theta_a & 0 \\
0 & 0 & \sin\theta_a & 0 \\
0 & 0 & 0 & 1
\end{bmatrix}
\]

\subsubsection*{3. Discretization}
To discretize the system with a sampling time \( \Delta T = 0.1 \), the discrete-time state-space equations are:
\[
x_{k+1} = A_d x_k + B_d u_k
\]
where:
\[
A_d \approx I + A \Delta T, \quad B_d \approx B \Delta T
\]

\noindent The discretized matrices are:
\[
A_d \approx
\begin{bmatrix}
1 & 0 & -v_g \sin\theta_g \Delta T & 0 & 0 & 0 \\
0 & 1 & v_g \cos\theta_g \Delta T & 0 & 0 & 0 \\
0 & 0 & 1 & 0 & 0 & 0 \\
0 & 0 & 0 & 1 & 0 & -v_a \sin\theta_a \Delta T \\
0 & 0 & 0 & 0 & 1 & v_a \cos\theta_a \Delta T \\
0 & 0 & 0 & 0 & 0 & 1
\end{bmatrix}, \quad
B_d \approx
\begin{bmatrix}
\cos\theta_g \Delta T & 0 & 0 & 0 \\
\sin\theta_g \Delta T & 0 & 0 & 0 \\
\frac{\tan\phi_g \Delta T}{L} & \frac{v_g \Delta T}{L \cos^2\phi_g} & 0 & 0 \\
0 & 0 & \cos\theta_a \Delta T & 0 \\
0 & 0 & \sin\theta_a \Delta T & 0 \\
0 & 0 & 0 & \Delta T
\end{bmatrix}
\]

\subsubsection*{4. Observability and Stability}
\begin{itemize}
    \item \textbf{Observability:} The observability matrix \( O \) is:
    \[
    O = \begin{bmatrix}
    C \\
    C A_d \\
    C A_d^2 \\
    \vdots \\
    C A_d^{n-1}
    \end{bmatrix}
    \]
    \( O \) must have full rank for the system to be observable. This depends on the measurement matrix \( C \).

    \item \textbf{Stability:} Stability is determined by the eigenvalues of \( A_d \). If all eigenvalues lie inside the unit circle (\( |\lambda| < 1 \)), the system is stable.
\end{itemize}

\subsubsection*{5. Summary}
\begin{itemize}
    \item The system is linearized about the nominal operating point \( x_0, u_0 \), and discrete-time matrices \( A_d, B_d \) are derived for \( \Delta T = 0.1 \).
    \item Observability depends on the rank of \( O \), and stability is determined by the eigenvalues of \( A_d \).
\end{itemize}

\subsection*{Exercise 3}


\subsubsection*{1. Objective}
The goal is to simulate the linearized discrete-time (DT) dynamics and measurement models near the linearization point, compare them with the full nonlinear system dynamics, and validate the accuracy of the linearized model. This is done by:
\begin{itemize}
    \item Simulating both the linearized DT model and the nonlinear model starting from a perturbed nominal initial condition.
    \item Assuming no process noise, measurement noise, or control input perturbations.
    \item Comparing the resulting states and measurements from both models through plots.
\end{itemize}

\subsubsection*{2. Simulation Setup}
The nominal state \( x_{\text{nom}}(t) \) is defined as a function of time for both the UGV and UAV:
\[
x_{\text{nom}}(t) = 
\begin{bmatrix}
\xi_{g,\text{nom}}(t) \\ \eta_{g,\text{nom}}(t) \\ \theta_{g,\text{nom}}(t) \\ 
\xi_{a,\text{nom}}(t) \\ \eta_{a,\text{nom}}(t) \\ \theta_{a,\text{nom}}(t)
\end{bmatrix}
\]
A small initial perturbation is added to the nominal state:
\[
\delta x_0 = [0.1, 0.1, 0.05, 0.1, 0.1, 0.05]^T, \quad x_{\text{perturbed}}(0) = x_{\text{nom}}(0) + \delta x_0
\]

\subsubsection*{3. Nonlinear Dynamics Simulation}
The nonlinear system is described by the following state equations:
\[
\dot{x} = f(x, u), \quad y = h(x)
\]
where:
\[
f(x, u) = 
\begin{bmatrix}
v_g \cos\theta_g \\ 
v_g \sin\theta_g \\ 
\frac{v_g}{L} \tan\phi_g \\ 
v_a \cos\theta_a \\ 
v_a \sin\theta_a \\ 
\omega_a
\end{bmatrix}, \quad
h(x) = \text{(measurement model)}
\]
The nonlinear system is simulated using `ode45` in MATLAB with the perturbed initial condition \( x_{\text{perturbed}}(0) \).

\subsubsection*{4. Linearized Dynamics Simulation}
The discrete-time linearized system is given by:
\[
\delta x_{k+1} = F_k \delta x_k + G_k \delta u_k, \quad \delta y_k = C_k \delta x_k
\]
where:
\[
F_k = I + A_k \Delta T, \quad G_k = B_k \Delta T
\]
The Jacobian matrices \( A_k \) and \( B_k \) are computed as:
\[
A_k = \frac{\partial f(x, u)}{\partial x}, \quad B_k = \frac{\partial f(x, u)}{\partial u}
\]
The linearized model is simulated for the same perturbed initial condition \( \delta x_0 \) and compared against the nonlinear simulation.

\subsubsection*{5. Measurement Comparison}
The measurement model is defined for range and bearing:
\[
C_k = 
\begin{bmatrix}
\frac{x_5 - x_2}{\sqrt{(x_5 - x_2)^2 + (x_4 - x_1)^2}} & \frac{-(x_4 - x_1)}{\sqrt{(x_5 - x_2)^2 + (x_4 - x_1)^2}} & 0 & \frac{-(x_5 - x_2)}{\sqrt{(x_5 - x_2)^2 + (x_4 - x_1)^2}} & \frac{x_4 - x_1}{\sqrt{(x_5 - x_2)^2 + (x_4 - x_1)^2}} & 0 \\
0 & 0 & 0 & 0 & 0 & 1
\end{bmatrix}
\]
Measurements \( \delta y_k \) from the linearized model are compared to the nonlinear measurements \( y_{\text{nonlinear}}(t) \) computed directly from the nonlinear simulation.

% \subsection*{6. Results and Validation}
% The states and measurements from the nonlinear model are compared with the linearized model. Key observations include:
% \begin{itemize}
%     \item The linearized DT model closely tracks the nonlinear system for small perturbations.
%     \item Deviations occur over time due to higher-order nonlinearities, especially for large perturbations.
%     \item The following plots summarize the results:
%     \begin{itemize}
%         \item \textbf{State Comparisons:} UGV states (\( \xi_g, \eta_g, \theta_g \)) and UAV states (\( \xi_a, \eta_a, \theta_a \)).
%         \item \textbf{Measurement Comparisons:} Range and bearing measurements.
%     \end{itemize}
% \end{itemize}


